!BOI

\newcommand{\aana}{{\em GEOS aana}}
\newcommand{\gsisa}{{\em GSI gridded component standalone}}
\newcommand{\gsi}{{\em GSI gridded component}}
\newcommand{\ptx}{{\tt protex}}

!  !TITLE: The \gsisa\ guide 

!  !AUTHORS: Carlos Cruz

!  !AFFILIATION: {Global Modeling and Assimilation Office, NASA/GSFC, Greenbelt, MD 20771}

!  !DATE: \today

!  !INTRODUCTION: Description of the \gsisa

\setlength{\parskip}{12pt}
%=========================================================================
\section{Introduction}

This document describes the \gsisa


%=========================================================================
\section{General Description of the \gsisa}

\subsection{A}

\subsection{B}

\begin{enumerate}

\item item1 
\item item2 

\end{enumerate}

\subsection{\em Example 2: HelloWorldMod}
  
\footnotesize
\begin{verbatim}
  module GSI
  end module GSI
\end{verbatim}
\normalsize


%.........................................................................
\newpage

{\Large \bf References}

\addcontentsline{toc}{section}{References} 

\begin{description}

\item DeLuca, C. and the ESMF Joint Specification Team, Earth System
Modeling Framework Project Plan 2005-2010, http://www.esmf.ucar.edu.

\item Hill, C., C. DeLuca, V. Balaji, M. Suarez, A. da Silva, The
Architecture of the Earth System Modeling Framework, {\em Computing in
Science and Engineering}, Vol. 11, No. 6, January/February 2004,
pp.18-28.

\item Held, I.M. and M.J. Suarez. A proposal for the intercomparison
  of the dynamical cores of atmospheric general circulation
  models. {\em Bulletin of the American Meteorological Society},
 {\bf 75(10)}:1825-1830, 1994.
 
\end{description} 

%..........................................................................


!EOI

